% Created 2022-04-29 Fr 21:49
% Intended LaTeX compiler: pdflatex
\documentclass[letterpaper, 11pt]{article}
\usepackage[utf8]{inputenc}
\usepackage[T1]{fontenc}
\usepackage{graphicx}
\usepackage{grffile}
\usepackage{longtable}
\usepackage{wrapfig}
\usepackage{rotating}
\usepackage[normalem]{ulem}
\usepackage{amsmath}
\usepackage{textcomp}
\usepackage{amssymb}
\usepackage{capt-of}
\usepackage{hyperref}
\usepackage{lmodern} % Ensures we have the right font
\usepackage[T1]{fontenc}
\usepackage[utf8]{inputenc}
\usepackage{graphicx, breqn}
\usepackage{mathtools, amsthm, amssymb, breqn}
\usepackage[table, xcdraw]{xcolor}
\definecolor{bblue}{HTML}{0645AD}
\usepackage[colorlinks]{hyperref}
\hypersetup{colorlinks, linkcolor=blue, urlcolor=bblue}
\usepackage{titling}
\setlength{\droptitle}{-6em}
\setlength{\parindent}{0pt}
\setlength{\parskip}{1em}
\usepackage[stretch=10]{microtype}
\usepackage{hyphenat}
\usepackage{ragged2e}
\usepackage{subfig} % Subfigures (not needed in Org I think)
\usepackage{hyperref} % Links
\usepackage{listings} % Code highlighting
\usepackage{pgfplots} % Plotting graphs
\usepackage{efbox}
\usepackage{tikz}
\usepackage{onimage} % note that onimage.sty must be in the working directory.
\usepackage[top=1in, bottom=1.25in, left=1.55in, right=1.55in]{geometry}
\renewcommand{\baselinestretch}{1.15}
\usepackage[explicit]{titlesec}
\pretitle{\begin{center}\fontsize{20pt}{20pt}\selectfont}
\posttitle{\par\end{center}}
\preauthor{\begin{center}\vspace{-6bp}\fontsize{14pt}{14pt}\selectfont}
\postauthor{\par\end{center}\vspace{-25bp}}
\predate{\begin{center}\fontsize{12pt}{12pt}\selectfont}
\postdate{\par\end{center}\vspace{0em}}
\titlespacing\section{0pt}{5pt}{5pt} % left margin, space before section header, space after section header
\titlespacing\subsection{0pt}{5pt}{-2pt} % left margin, space before subsection header, space after subsection header
\titlespacing\subsubsection{0pt}{5pt}{-2pt} % left margin, space before subsection header, space after subsection header
\usepackage{enumitem}
\setlist{itemsep=-2pt} % or \setlist{noitemsep} to leave space around whole list
\usepackage {mathtools , amssymb , amsthm}
\author{Fred Mitchell}
\date{\today}
\title{Slit Scan Notes}
\hypersetup{
 pdfauthor={Fred Mitchell},
 pdftitle={Slit Scan Notes},
 pdfkeywords={},
 pdfsubject={},
 pdfcreator={Emacs 28.1 (Org mode 9.1.5)}, 
 pdflang={English}}
\begin{document}

\maketitle
\tableofcontents


\section{Slit Scan Notes}
\label{sec:org3a63403}
\subsection{Disclaimer}
\label{sec:orge4d2be4}
These are my basic notes on this project, and are not
meant for general consumption, and therefore are not
gauranteed to be accurate or even useful for anyone 
other than myself. Please see the README.org or the
README.pdf for that.
\subsection{Loading Images}
\label{sec:org83330a2}
\begin{itemize}
\item We could resize one of the images to be the 
exact same size/dimensions of the other. Would simplify 
the math quite a bit. But might cause some wierdness
in some circumstances. Of course, the user
can correct the issue and make sure both images
are the same dimensions, etc.
\item Trying to determine how to transverse the images 
slit-wise. In mapping the images, we are simply
given the x,y coordinates. Of course, we can
work backwards from there to determine the "slit",
and should cover this in the math description in the README.
\end{itemize}


\subsection{Outputting the video}
\label{sec:orgdd0bc43}
\subsection{Future planned enhancements}
\label{sec:orgcbba795}
\end{document}